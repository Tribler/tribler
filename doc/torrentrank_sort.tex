\documentclass[a4paper,11pt]{article}
\usepackage[a4paper]{geometry}

\title{Heuristic algorithm for ranking torrent search results}
\author{Steffan Norberhuis \and Quinten Stokkink}
\date{27 May 2013}
\begin{document}
   \maketitle
\section{Introduction}
Our Tribler extension needs to handle (movie) torrent search result ranking based on some complex
parameters. Given that there is no perfect way to order torrents based on more than just
seeders and leechers (taking into account user preferences and such) we have produced a
heuristic algorithm to handle this sorting. This document will describe the choices that were
made in designing this algorithm.

\section{Inputs}
The inputs for our algorithm are the following ones:
\begin{itemize}
\item USER\_RATING: A user rating for a provider, a floating point number ranging from $0.0$ to $1.0$ 
\item NUM\_LEECH: The number of leechers for a torrent, an integer $\ge0$
\item NUM\_SEEDS: The number of seeders for a torrent, an integer $\ge0$
\item NAME\_MATCH: The matching of the torrent name compared to the movie details, a floating point number $\ge 0.0$
\item SPEC\_TERMS: The matching of special terms, set by the user, compared to the torrent name, a floating point num ber $\ge 0.0$
\end{itemize}

\textbf{USER\_RATING}\\
The user rating is given to a torrent search provider by the user.
This allows the user to prioritize sites they trust.
For example one might rank torrents from $www.website.com$ better than torrents retrieved from
$www.shady.com$. In this example the algorithm should choose a torrent from $www.website.com$
over a comparable one from $www.shady.com$.
The user can specify this preference using a floating point number on the interval from 0.0 to 1.0,
corresponding to a certain trust percentage.

\textbf{NUM\_LEECH}\\
The number of leechers for a torrent.
This is an integer larger or equal to 0.

\textbf{NUM\_SEEDS}\\
The number of seeders for a torrent.
This is an integer larger or equal to 0.

\textbf{NAME\_MATCH}\\
This number specifies the summation of all the fuzzy matches between the torrent name and the parameters
of the movie we searched for. For example $(MovieName, Directors)$ matches ``MovieNameDirectors'' for a total
of $100\%$ and ``MovieNamer'' for $50\%$.

\textbf{SPEC\_TERMS}\\
The user can also specify terms that signify some sort of quality. 
For example if we encounter ''TheQualityReleaseGroup`` in a torrent title and the user has marked
this word as being an indicator of quality, the search algorithm should rank this result higher than
other search results.

\section{Algorithm}
What the algorithm must do is provide some sort of rank per search result.
We choose the format of ``the higher the rank, the better the search result''.
To produce this rank we will calculate two sub-results and then use these sub-results to produce a final rank.
The sub-results are:
\begin{itemize}
\item POT\_SPEED: The potential download speed we could get from a torrent, a floating point number $\ge0.0$
\item TECH\_WANT: The combination of how much the torrent name matches the search and special term
criteria, a floating point number $\ge1.0$
\end{itemize}

\textbf{POT\_SPEED}\\
To estimate the speed of a torrent we look at the number of seeders and leechers it has.
We also note that leechers inherently provide slower download speeds to us than seeders do.
As a rule of thumb, we estimate a leecher gives us about half the speed of a seeder.
What this actual speed is, is of no importance.
The potential speed (expressed in units of avarage download speed from seeders for an arbitrary torrent)
then becomes:\\
$POT\_SPEED = NUM\_SEEDS + 0.5 \times NUM\_LEECH$

\textbf{TECH\_WANT}\\
To estimate the relevance of a torrent we base ourselves on the user's search criteria and special terms.
We assume a good torrent is well named and specifies some quality identifiers in its name.
This matching to our criteria will weigh any potential download speed.
We assume that a name match is worth as much as a special term.
Combining weights (and asserting we never end up with a weight of 0, because there are no special terms
or name matches) we end up with the following formula:\\
$TECH\_WANT = (SPEC\_TERMS + 1) * (NAME\_MATCH + 1)$\\
Note that if multiplied with the download speed this will be equivalent to first weighing with the 
name matching rate and then weighing with the special term matching rate.

\textbf{RANK}\\
We can now finally calculate a rank for a torrent.
If we multiply the relevance of the torrent to the potential speed we can get we get an estimate for how
much we want this torrent. For example a torrent with extremely high download speed but a slightly
dodgy name will beat a torrent that ``looks good'' but provides little to no download speed.
Lastly we also want to factor in how much a user cares for a certain provider.
If a user knows that, however popular a torrent may be, a certain site produces bad torrents mainly, the
algorithm should downgrade the search result.
The final ranking algorithm then becomes:\\
$RANK = USER\_RATING \times TECH\_WANT \times POT\_SPEED$
\end{document}